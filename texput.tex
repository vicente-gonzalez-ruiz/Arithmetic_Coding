%%% Local Variables:
%%% mode: latex
%%% TeX-master: "<none>"
%%% End:

\title{Arithmetic coding}

\author{Vicente González Ruiz}

\maketitle

\section{Basics}

\begin{itemize}
\tightlist
\item
  Arithmetic coding~\cite{rissanen1979arithmetic, witten1987arithmetic}relaxes the Eq. (Eq:Huffman), verifying that, for
  every encoded symbol, \begin{equation}
    l\big(c(s)\big) = I(s), \tag{Eq:arithmetic}
  \end{equation} i.e.~the number of bits of data (code-word) assigned by
  the encoder is equal to the number of bits of information that the
  symbol represent.
\end{itemize}

\img{600}{graphics/comparacion.png}

\begin{itemize}
\tightlist
\item
  It can be said that, arithmetic coding is optimal because the average
  length of an arithmetic code is equal to the entropy of the
  information source, measured in bits/symbol.
\end{itemize}
\img{500}{graphics/aritmetica_ejemplo.png}

\section{An ideal encoder}
\begin{enumerate}
\def\labelenumi{\arabic{enumi}.}
\tightlist
\item
  Let \([L,H)\leftarrow [0.0,1.0)\) an interval of real numbers.
\item
  While the input is not exhausted:
  \begin{enumerate}
  \def\labelenumii{\arabic{enumii}.}
  \tightlist
  \item
    Split \([L,H)\) into so many sub-intervals as different symbols are
    in the alphabet. The size of each sub-interval is proportional to
    the probability of the corresponding symbol.
  \item
    Select the sub-interval \([L',H')\) associated with the encoded
    symbol.
  \item
    \([L,H)\leftarrow [L',H')\).
  \end{enumerate}
\item
  Output a real number \(x\in[L,H)\) (the arithmetic code-stream). The
  number of decimals of \(x\) should be large enough to distinguish the
  final sub-interval \([L,H)\) from the rest of possibilities.
\end{enumerate}

\subsection{Example}
\begin{itemize}
\tightlist
\item
  Imagine a binary sequence, where \(p(\text{A})=3/4\) and
  \(p(\text{B})=1/4\). Compute the arithmetic code of the sequences A,
  B, AA, AB, BA y BB.
\end{itemize}

\section{An ideal decoder}
\begin{enumerate}
\def\labelenumi{\arabic{enumi}.}
\tightlist
\item
  Let \([L,H)\leftarrow [0.0,1.0)\) the initial interval.
\item
  While the input is not exhausted:
  \begin{enumerate}
  \def\labelenumii{\arabic{enumii}.}
  \tightlist
  \item
    Split \([L,H)\) in so many sub-intervals as different symbols are in
    the alphabet. The size of each sub-interval is proportional to the
    probability of the corresponding symbol.
  \item
    Input so many bits of \(x\) as they are needed to:

    \begin{enumerate}
    \def\labelenumiii{\arabic{enumiii}.}
    \tightlist
    \item
      Select the sub-interval \([L',H')\) that contains \(x\).
    \item
      Output the symbol that \([L',H')\) represents.
    \item
      \([L,H)\leftarrow[L',H')\).
    \end{enumerate}
  \end{enumerate}
\end{enumerate}

\subsection{Example}

TO-DO

\section{Incremental transmission}
\begin{itemize}
\item
  It is not necessary to wait for the end of the encoding to generate
  the arithmetic code. When we work with binary representations of the
  real numbers \(L\) and \(H\), their most significant bits become
  identical when the interval is reduced. These bits belong to the
  output arithmetic code, therefore, they can be output as soon as they
  match.

  For example, when the symbol B is encoded, a code-bit 1 can be output
  because any sequence of symbols that start with B have a code-word
  that begins with 1.
\item
  When the most significant bits of \(L\) and \(H\) are output, the bits
  of each register are shifted to the left, and new bits need to be
  inserted. The results is an automatic zoom of the selected
  sub-interval.

  Following with the previous example, the register shifting generates
  an ampliation of the \([0.50,1.00)\) interval to the \([0.00,1.00)\).
\end{itemize}

\section{Lab}

TO-DO.

\bibliography{text-compression}
